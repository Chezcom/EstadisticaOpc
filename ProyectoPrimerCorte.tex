% Options for packages loaded elsewhere
\PassOptionsToPackage{unicode}{hyperref}
\PassOptionsToPackage{hyphens}{url}
%
\documentclass[
]{article}
\usepackage{amsmath,amssymb}
\usepackage{iftex}
\ifPDFTeX
  \usepackage[T1]{fontenc}
  \usepackage[utf8]{inputenc}
  \usepackage{textcomp} % provide euro and other symbols
\else % if luatex or xetex
  \usepackage{unicode-math} % this also loads fontspec
  \defaultfontfeatures{Scale=MatchLowercase}
  \defaultfontfeatures[\rmfamily]{Ligatures=TeX,Scale=1}
\fi
\usepackage{lmodern}
\ifPDFTeX\else
  % xetex/luatex font selection
    \setmainfont[]{DejaVu Sans}
\fi
% Use upquote if available, for straight quotes in verbatim environments
\IfFileExists{upquote.sty}{\usepackage{upquote}}{}
\IfFileExists{microtype.sty}{% use microtype if available
  \usepackage[]{microtype}
  \UseMicrotypeSet[protrusion]{basicmath} % disable protrusion for tt fonts
}{}
\makeatletter
\@ifundefined{KOMAClassName}{% if non-KOMA class
  \IfFileExists{parskip.sty}{%
    \usepackage{parskip}
  }{% else
    \setlength{\parindent}{0pt}
    \setlength{\parskip}{6pt plus 2pt minus 1pt}}
}{% if KOMA class
  \KOMAoptions{parskip=half}}
\makeatother
\usepackage{xcolor}
\usepackage[margin=1in]{geometry}
\usepackage{color}
\usepackage{fancyvrb}
\newcommand{\VerbBar}{|}
\newcommand{\VERB}{\Verb[commandchars=\\\{\}]}
\DefineVerbatimEnvironment{Highlighting}{Verbatim}{commandchars=\\\{\}}
% Add ',fontsize=\small' for more characters per line
\usepackage{framed}
\definecolor{shadecolor}{RGB}{248,248,248}
\newenvironment{Shaded}{\begin{snugshade}}{\end{snugshade}}
\newcommand{\AlertTok}[1]{\textcolor[rgb]{0.94,0.16,0.16}{#1}}
\newcommand{\AnnotationTok}[1]{\textcolor[rgb]{0.56,0.35,0.01}{\textbf{\textit{#1}}}}
\newcommand{\AttributeTok}[1]{\textcolor[rgb]{0.13,0.29,0.53}{#1}}
\newcommand{\BaseNTok}[1]{\textcolor[rgb]{0.00,0.00,0.81}{#1}}
\newcommand{\BuiltInTok}[1]{#1}
\newcommand{\CharTok}[1]{\textcolor[rgb]{0.31,0.60,0.02}{#1}}
\newcommand{\CommentTok}[1]{\textcolor[rgb]{0.56,0.35,0.01}{\textit{#1}}}
\newcommand{\CommentVarTok}[1]{\textcolor[rgb]{0.56,0.35,0.01}{\textbf{\textit{#1}}}}
\newcommand{\ConstantTok}[1]{\textcolor[rgb]{0.56,0.35,0.01}{#1}}
\newcommand{\ControlFlowTok}[1]{\textcolor[rgb]{0.13,0.29,0.53}{\textbf{#1}}}
\newcommand{\DataTypeTok}[1]{\textcolor[rgb]{0.13,0.29,0.53}{#1}}
\newcommand{\DecValTok}[1]{\textcolor[rgb]{0.00,0.00,0.81}{#1}}
\newcommand{\DocumentationTok}[1]{\textcolor[rgb]{0.56,0.35,0.01}{\textbf{\textit{#1}}}}
\newcommand{\ErrorTok}[1]{\textcolor[rgb]{0.64,0.00,0.00}{\textbf{#1}}}
\newcommand{\ExtensionTok}[1]{#1}
\newcommand{\FloatTok}[1]{\textcolor[rgb]{0.00,0.00,0.81}{#1}}
\newcommand{\FunctionTok}[1]{\textcolor[rgb]{0.13,0.29,0.53}{\textbf{#1}}}
\newcommand{\ImportTok}[1]{#1}
\newcommand{\InformationTok}[1]{\textcolor[rgb]{0.56,0.35,0.01}{\textbf{\textit{#1}}}}
\newcommand{\KeywordTok}[1]{\textcolor[rgb]{0.13,0.29,0.53}{\textbf{#1}}}
\newcommand{\NormalTok}[1]{#1}
\newcommand{\OperatorTok}[1]{\textcolor[rgb]{0.81,0.36,0.00}{\textbf{#1}}}
\newcommand{\OtherTok}[1]{\textcolor[rgb]{0.56,0.35,0.01}{#1}}
\newcommand{\PreprocessorTok}[1]{\textcolor[rgb]{0.56,0.35,0.01}{\textit{#1}}}
\newcommand{\RegionMarkerTok}[1]{#1}
\newcommand{\SpecialCharTok}[1]{\textcolor[rgb]{0.81,0.36,0.00}{\textbf{#1}}}
\newcommand{\SpecialStringTok}[1]{\textcolor[rgb]{0.31,0.60,0.02}{#1}}
\newcommand{\StringTok}[1]{\textcolor[rgb]{0.31,0.60,0.02}{#1}}
\newcommand{\VariableTok}[1]{\textcolor[rgb]{0.00,0.00,0.00}{#1}}
\newcommand{\VerbatimStringTok}[1]{\textcolor[rgb]{0.31,0.60,0.02}{#1}}
\newcommand{\WarningTok}[1]{\textcolor[rgb]{0.56,0.35,0.01}{\textbf{\textit{#1}}}}
\usepackage{graphicx}
\makeatletter
\def\maxwidth{\ifdim\Gin@nat@width>\linewidth\linewidth\else\Gin@nat@width\fi}
\def\maxheight{\ifdim\Gin@nat@height>\textheight\textheight\else\Gin@nat@height\fi}
\makeatother
% Scale images if necessary, so that they will not overflow the page
% margins by default, and it is still possible to overwrite the defaults
% using explicit options in \includegraphics[width, height, ...]{}
\setkeys{Gin}{width=\maxwidth,height=\maxheight,keepaspectratio}
% Set default figure placement to htbp
\makeatletter
\def\fps@figure{htbp}
\makeatother
\setlength{\emergencystretch}{3em} % prevent overfull lines
\providecommand{\tightlist}{%
  \setlength{\itemsep}{0pt}\setlength{\parskip}{0pt}}
\setcounter{secnumdepth}{-\maxdimen} % remove section numbering
\ifLuaTeX
  \usepackage{selnolig}  % disable illegal ligatures
\fi
\usepackage{bookmark}
\IfFileExists{xurl.sty}{\usepackage{xurl}}{} % add URL line breaks if available
\urlstyle{same}
\hypersetup{
  pdftitle={Fundamentos de Estadística para Analítica de datos},
  hidelinks,
  pdfcreator={LaTeX via pandoc}}

\title{Fundamentos de Estadística para Analítica de datos}
\author{}
\date{\vspace{-2.5em}}

\begin{document}
\maketitle

\subsection{Fundamentos de Estadística para Analítica de
datos}\label{fundamentos-de-estaduxedstica-para-analuxedtica-de-datos}

Integrantes

\begin{itemize}
\tightlist
\item
  Jimmy José Díaz Leyton
\item
  Jessica Hasbleidy Pinilla Segovia
\item
  Yenny Marisol Sánchez Sánchez
\end{itemize}

El uso de vehículos eléctricos ha experimentado un crecimiento
significativo en los últimos años debido al interés en reducir la
dependencia de los combustibles fósiles. Por ello, se ha tomado un
conjunto de datos de vehículos eléctricos registrados en el estado de
Washington con el propósito de examinar distintos aspectos relacionados
con su comercialización y desempeño.

Se analizarán variables como el tipo de vehículo eléctrico (BEV -
totalmente eléctrico y PHEV - híbrido enchufable), el precio base (MSRP)
y el alcance eléctrico (Electric Range), con el fin de identificar
patrones y diferencias entre estos modelos.

Además, se explorará la evolución de los precios a lo largo del tiempo,
la relación entre la autonomía y el tipo de vehículo, así como la
influencia de factores geográficos en la elegibilidad para incentivos de
energía limpia. A través de este análisis, se espera obtener información
valiosa sobre la adopción y el desarrollo de los vehículos eléctricos en
Washington, contribuyendo a una mejor comprensión de su impacto en el
mercado y en la infraestructura de transporte sostenible.

\subsubsection{\texorpdfstring{\textbf{Variables
cualitativas:}}{Variables cualitativas:}}\label{variables-cualitativas}

\begin{enumerate}
\def\labelenumi{\arabic{enumi}.}
\item
  \textbf{City} (Ciudad) - \textbf{Nominal}
\item
  \textbf{State} (Estado) - \textbf{Nominal}: ``Washington''.
\item
  \textbf{Make} (Marca) - \textbf{Nominal}: Diferentes marcas de
  vehículos, como Tesla, Nissan, etc.
\item
  \textbf{Model} (Modelo) - \textbf{Nominal}: Diferentes modelos de
  vehículos.
\item
  \textbf{Electric Vehicle Type} (Tipo de vehículo eléctrico) -
  \textbf{Nominal}: Puede ser BEV (Battery Electric Vehicle) o PHEV
  (Plug-in Hybrid Electric Vehicle).
\item
  \textbf{Clean Alternative Fuel Vehicle (CAFV) Eligibility}
  (Preferencia para vehículo de combustible alternativo limpio) -
  \textbf{Nominal}
\item
  \textbf{Vehicle Location} (Ubicación del vehículo) - \textbf{Nominal}
\item
  \textbf{Electric Utility} (Utilidad eléctrica) - \textbf{Nominal}: Es
  el nombre de la empresa de servicios eléctricos que abastece el área.
\item
  \textbf{Legislative District} (Distrito legislativo) -
  \textbf{Nominal}: Representa los diferentes distritos legislativos en
  el estado de Washington.
\end{enumerate}

\subsubsection{\texorpdfstring{\textbf{Variables
cuantitativas:}}{Variables cuantitativas:}}\label{variables-cuantitativas}

\begin{enumerate}
\def\labelenumi{\arabic{enumi}.}
\item
  \textbf{Postal Code} (Código Postal) - \textbf{Discreta}: Es un número
  que representa un código de área específico, pero es una variable
  discreta porque los códigos postales están en un conjunto limitado de
  valores.
\item
  \textbf{Model Year} (Año del modelo) - \textbf{Discreta}: Es un número
  entero que indica el año en que se fabricó el vehículo. Aunque es
  numérica, es discreta porque los valores son enteros y finitos.
\item
  \textbf{Electric Range} (Alcance eléctrico) - \textbf{Continua}: Es la
  distancia que puede recorrer un vehículo con una sola carga. Como
  puede tomar cualquier valor real dentro de un rango, es una variable
  continua.
\item
  \textbf{Base MSRP} (Precio base MSRP) - \textbf{Continua}: Es el
  precio base del vehículo, que puede tomar cualquier valor dentro de un
  rango, lo que lo convierte en una variable continua.
\item
  \textbf{2020 Census Tract} (Tracto censal de 2020) - \textbf{Nominal o
  discreta}: Aunque parece ser un identificador geográfico, en algunos
  casos puede ser tratado como una variable nominal (es un código de
  área), pero dependiendo del análisis, puede también considerarse
  discreta si se le usa para representar zonas geográficas.
\end{enumerate}

\subsubsection{\texorpdfstring{\textbf{Pregunta
global:}}{Pregunta global:}}\label{pregunta-global}

\textbf{¿Cuál es la relación entre el tipo de vehículo eléctrico (BEV vs
PHEV), el precio base (MSRP) y el alcance eléctrico (Electric Range) de
los vehículos registrados en el Estado de Washington?}

Esta pregunta abarca una perspectiva amplia que examina las diferencias
entre los tipos de vehículos eléctricos (BEV y PHEV) en relación con dos
variables cuantitativas clave: el precio y el alcance eléctrico. Esto te
permitirá ver si existen patrones o diferencias significativas entre
estos dos tipos de vehículos en cuanto a su precio y rendimiento.

\subsubsection{\texorpdfstring{\textbf{Preguntas adicionales para apoyar
la
respuesta:}}{Preguntas adicionales para apoyar la respuesta:}}\label{preguntas-adicionales-para-apoyar-la-respuesta}

\begin{enumerate}
\def\labelenumi{\arabic{enumi}.}
\item
  \textbf{¿Cómo varía el precio base (MSRP) de los vehículos eléctricos
  en función del año del modelo?\\
  }

  Esta pregunta te permitirá analizar cómo ha cambiado el precio a lo
  largo de los años. Esto es útil para entender si los vehículos más
  nuevos tienden a ser más caros y si esto está relacionado con el
  avance de la tecnología.
\item
  \textbf{¿Existen diferencias en el alcance eléctrico (Electric Range)
  entre los vehículos BEV y los PHEV?\\
  \strut \\
  }Este análisis podría ayudar a determinar si los vehículos
  completamente eléctricos (BEV) tienen un mayor alcance en comparación
  con los híbridos enchufables (PHEV) o si no hay diferencias
  significativas.
\item
  \textbf{¿Cómo varía la elegibilidad para los vehículos de combustible
  alternativo limpio (CAFV) en función del tipo de vehículo y su
  ubicación geográfica (estado o ciudad)?\\
  }

  Esta pregunta te permitiría explorar si hay diferencias en la
  elegibilidad para incentivos o beneficios dependiendo del tipo de
  vehículo (BEV o PHEV) y cómo esto se relaciona con la ciudad o región
  en la que están registrados los vehículos.
\item
  \textbf{¿Qué relación existe entre el tipo de vehículo (BEV o PHEV) y
  la utilidad eléctrica que abastece la región en la que está registrado
  el vehículo?\\
  \strut \\
  }Esta pregunta busca explorar si existe alguna correlación entre el
  tipo de vehículo eléctrico y el proveedor de electricidad que tiene la
  región, lo que podría influir en la accesibilidad y la infraestructura
  de carga disponible.
\end{enumerate}

\subsubsection{Cargar librerías
necesarias}\label{cargar-libreruxedas-necesarias}

\begin{Shaded}
\begin{Highlighting}[]
\FunctionTok{library}\NormalTok{(tidyverse)}
\end{Highlighting}
\end{Shaded}

\begin{verbatim}
## Warning: package 'tidyverse' was built under R version 4.4.3
\end{verbatim}

\begin{verbatim}
## Warning: package 'lubridate' was built under R version 4.4.3
\end{verbatim}

\begin{verbatim}
## -- Attaching core tidyverse packages ------------------------ tidyverse 2.0.0 --
## v dplyr     1.1.4     v readr     2.1.5
## v forcats   1.0.0     v stringr   1.5.1
## v ggplot2   3.5.1     v tibble    3.2.1
## v lubridate 1.9.4     v tidyr     1.3.1
## v purrr     1.0.2     
## -- Conflicts ------------------------------------------ tidyverse_conflicts() --
## x dplyr::filter() masks stats::filter()
## x dplyr::lag()    masks stats::lag()
## i Use the conflicted package (<http://conflicted.r-lib.org/>) to force all conflicts to become errors
\end{verbatim}

\begin{Shaded}
\begin{Highlighting}[]
\FunctionTok{library}\NormalTok{(dplyr)}
\end{Highlighting}
\end{Shaded}

\subsubsection{Cargar datos (ajusta el nombre del
archivo)}\label{cargar-datos-ajusta-el-nombre-del-archivo}

\begin{Shaded}
\begin{Highlighting}[]
\NormalTok{datos}\OtherTok{\textless{}{-}} \FunctionTok{read.csv}\NormalTok{(}\StringTok{"C:/Users/OptiPlex/OneDrive/Documentos/Universidad\_Central/Opcion Estadistica/Proyecto/Electric\_Vehicle\_Population\_Data.csv"}\NormalTok{)}
\end{Highlighting}
\end{Shaded}

\subsubsection{Ver las primeras filas}\label{ver-las-primeras-filas}

\begin{Shaded}
\begin{Highlighting}[]
\FunctionTok{head}\NormalTok{(datos)}
\end{Highlighting}
\end{Shaded}

\begin{verbatim}
##   VIN..1.10.   County    City State Postal.Code Model.Year    Make      Model
## 1 5YJ3E1EBXK     King Seattle    WA       98178       2019   TESLA    MODEL 3
## 2 5YJYGDEE3L   Kitsap Poulsbo    WA       98370       2020   TESLA    MODEL Y
## 3 KM8KRDAF5P   Kitsap  Olalla    WA       98359       2023 HYUNDAI    IONIQ 5
## 4 5UXTA6C0XM   Kitsap Seabeck    WA       98380       2021     BMW         X5
## 5 JTMAB3FV7P Thurston Rainier    WA       98576       2023  TOYOTA RAV4 PRIME
## 6 5YJSA1DN0C Thurston Olympia    WA       98502       2012   TESLA    MODEL S
##                    Electric.Vehicle.Type
## 1         Battery Electric Vehicle (BEV)
## 2         Battery Electric Vehicle (BEV)
## 3         Battery Electric Vehicle (BEV)
## 4 Plug-in Hybrid Electric Vehicle (PHEV)
## 5 Plug-in Hybrid Electric Vehicle (PHEV)
## 6         Battery Electric Vehicle (BEV)
##              Clean.Alternative.Fuel.Vehicle..CAFV..Eligibility Electric.Range
## 1                      Clean Alternative Fuel Vehicle Eligible            220
## 2                      Clean Alternative Fuel Vehicle Eligible            291
## 3 Eligibility unknown as battery range has not been researched              0
## 4                      Clean Alternative Fuel Vehicle Eligible             30
## 5                      Clean Alternative Fuel Vehicle Eligible             42
## 6                      Clean Alternative Fuel Vehicle Eligible            265
##   Base.MSRP Legislative.District DOL.Vehicle.ID            Vehicle.Location
## 1         0                   37      477309682 POINT (-122.23825 47.49461)
## 2         0                   23      109705683 POINT (-122.64681 47.73689)
## 3         0                   26      230390492 POINT (-122.54729 47.42602)
## 4         0                   35      267929112 POINT (-122.81585 47.64509)
## 5         0                    2      236505139 POINT (-122.68993 46.88897)
## 6     59900                   22      186637195 POINT (-122.92333 47.03779)
##                               Electric.Utility X2020.Census.Tract
## 1 CITY OF SEATTLE - (WA)|CITY OF TACOMA - (WA)        53033011902
## 2                       PUGET SOUND ENERGY INC        53035091100
## 3                       PUGET SOUND ENERGY INC        53035092802
## 4                       PUGET SOUND ENERGY INC        53035091301
## 5                       PUGET SOUND ENERGY INC        53067012530
## 6                       PUGET SOUND ENERGY INC        53067010600
\end{verbatim}

\subsubsection{Ver estructura de los
datos}\label{ver-estructura-de-los-datos}

\begin{Shaded}
\begin{Highlighting}[]
\FunctionTok{str}\NormalTok{(datos)}
\end{Highlighting}
\end{Shaded}

\begin{verbatim}
## 'data.frame':    235692 obs. of  17 variables:
##  $ VIN..1.10.                                       : chr  "5YJ3E1EBXK" "5YJYGDEE3L" "KM8KRDAF5P" "5UXTA6C0XM" ...
##  $ County                                           : chr  "King" "Kitsap" "Kitsap" "Kitsap" ...
##  $ City                                             : chr  "Seattle" "Poulsbo" "Olalla" "Seabeck" ...
##  $ State                                            : chr  "WA" "WA" "WA" "WA" ...
##  $ Postal.Code                                      : int  98178 98370 98359 98380 98576 98502 98004 98271 98034 98052 ...
##  $ Model.Year                                       : int  2019 2020 2023 2021 2023 2012 2017 2022 2018 2018 ...
##  $ Make                                             : chr  "TESLA" "TESLA" "HYUNDAI" "BMW" ...
##  $ Model                                            : chr  "MODEL 3" "MODEL Y" "IONIQ 5" "X5" ...
##  $ Electric.Vehicle.Type                            : chr  "Battery Electric Vehicle (BEV)" "Battery Electric Vehicle (BEV)" "Battery Electric Vehicle (BEV)" "Plug-in Hybrid Electric Vehicle (PHEV)" ...
##  $ Clean.Alternative.Fuel.Vehicle..CAFV..Eligibility: chr  "Clean Alternative Fuel Vehicle Eligible" "Clean Alternative Fuel Vehicle Eligible" "Eligibility unknown as battery range has not been researched" "Clean Alternative Fuel Vehicle Eligible" ...
##  $ Electric.Range                                   : int  220 291 0 30 42 265 81 22 215 215 ...
##  $ Base.MSRP                                        : int  0 0 0 0 0 59900 0 0 0 0 ...
##  $ Legislative.District                             : int  37 23 26 35 2 22 48 39 45 45 ...
##  $ DOL.Vehicle.ID                                   : int  477309682 109705683 230390492 267929112 236505139 186637195 196789610 204822761 2039222 474817283 ...
##  $ Vehicle.Location                                 : chr  "POINT (-122.23825 47.49461)" "POINT (-122.64681 47.73689)" "POINT (-122.54729 47.42602)" "POINT (-122.81585 47.64509)" ...
##  $ Electric.Utility                                 : chr  "CITY OF SEATTLE - (WA)|CITY OF TACOMA - (WA)" "PUGET SOUND ENERGY INC" "PUGET SOUND ENERGY INC" "PUGET SOUND ENERGY INC" ...
##  $ X2020.Census.Tract                               : num  5.30e+10 5.30e+10 5.30e+10 5.30e+10 5.31e+10 ...
\end{verbatim}

\subsubsection{Pregunta 1:}\label{pregunta-1}

Filtrar datos para BEV y PHEV

\begin{Shaded}
\begin{Highlighting}[]
\NormalTok{vehiculos\_filtrados }\OtherTok{\textless{}{-}}\NormalTok{ datos }\SpecialCharTok{\%\textgreater{}\%}
  \FunctionTok{filter}\NormalTok{(Electric.Vehicle.Type }\SpecialCharTok{\%in\%} \FunctionTok{c}\NormalTok{(}\StringTok{"Battery Electric Vehicle (BEV)"}\NormalTok{, }\StringTok{"Plug{-}in Hybrid Electric Vehicle (PHEV)"}\NormalTok{))}
\end{Highlighting}
\end{Shaded}

\subsubsection{Crear el gráfico}\label{crear-el-gruxe1fico}

\begin{Shaded}
\begin{Highlighting}[]
\CommentTok{\#windows()}
\FunctionTok{ggplot}\NormalTok{(vehiculos\_filtrados, }\FunctionTok{aes}\NormalTok{(}\AttributeTok{x =}\NormalTok{ Electric.Vehicle.Type, }\AttributeTok{fill =}\NormalTok{ Electric.Vehicle.Type)) }\SpecialCharTok{+}
  \FunctionTok{geom\_bar}\NormalTok{() }\SpecialCharTok{+}
  \FunctionTok{labs}\NormalTok{(}\AttributeTok{title =} \StringTok{"Distribución de Vehículos Eléctricos (BEV vs PHEV)"}\NormalTok{,}
       \AttributeTok{x =} \StringTok{"Tipo de Vehículo"}\NormalTok{,}\AttributeTok{y =} \StringTok{"Cantidad"}\NormalTok{,}\AttributeTok{fill =} \StringTok{"Tipo"}\NormalTok{) }\SpecialCharTok{+}  \FunctionTok{theme\_minimal}\NormalTok{()}
\end{Highlighting}
\end{Shaded}

\includegraphics{ProyectoPrimerCorte_files/figure-latex/unnamed-chunk-6-1.pdf}

\#\#\#Pregunta 3: Diferencias en el alcance eléctrico entre BEV y PHEV

\subsubsection{Gráfico de boxplot comparando BEV y
PHEV}\label{gruxe1fico-de-boxplot-comparando-bev-y-phev}

\begin{Shaded}
\begin{Highlighting}[]
\FunctionTok{ggplot}\NormalTok{(datos, }\FunctionTok{aes}\NormalTok{(}\AttributeTok{x =}\NormalTok{ Electric.Vehicle.Type, }\AttributeTok{y =}\NormalTok{ Electric.Range, }\AttributeTok{fill =}
\NormalTok{Electric.Vehicle.Type)) }\SpecialCharTok{+} \FunctionTok{geom\_boxplot}\NormalTok{() }\SpecialCharTok{+} \FunctionTok{labs}\NormalTok{(}\AttributeTok{title =} \StringTok{"Comparación del}
\StringTok{Alcance entre BEV y PHEV"}\NormalTok{, }\AttributeTok{x =} \StringTok{"Tipo de Vehículo"}\NormalTok{, }\AttributeTok{y =} \StringTok{"Alcance}
\StringTok{Eléctrico (millas)"}\NormalTok{) }\SpecialCharTok{+} \FunctionTok{theme\_minimal}\NormalTok{()}
\end{Highlighting}
\end{Shaded}

\begin{verbatim}
## Warning: Removed 36 rows containing non-finite outside the scale range
## (`stat_boxplot()`).
\end{verbatim}

\includegraphics{ProyectoPrimerCorte_files/figure-latex/unnamed-chunk-7-1.pdf}

\#\#\#Pregunta 4: Relación entre elegibilidad CAFV y ubicación

\subsubsection{Contar registros por ciudad y
elegibilidad}\label{contar-registros-por-ciudad-y-elegibilidad}

\begin{Shaded}
\begin{Highlighting}[]
\NormalTok{ciudad\_elegibilidad }\OtherTok{\textless{}{-}}\NormalTok{ datos }\SpecialCharTok{\%\textgreater{}\%}
  \FunctionTok{group\_by}\NormalTok{(City, Clean.Alternative.Fuel.Vehicle..CAFV..Eligibility) }\SpecialCharTok{\%\textgreater{}\%}
  \FunctionTok{summarise}\NormalTok{(}\AttributeTok{Conteo =} \FunctionTok{n}\NormalTok{(), }\AttributeTok{.groups =} \StringTok{"drop"}\NormalTok{) }\SpecialCharTok{\%\textgreater{}\%}
  \FunctionTok{arrange}\NormalTok{(}\FunctionTok{desc}\NormalTok{(Conteo))}
\end{Highlighting}
\end{Shaded}

\subsubsection{Obtener la moda (ciudades con mayor frecuencia por tipo
de
elegibilidad)}\label{obtener-la-moda-ciudades-con-mayor-frecuencia-por-tipo-de-elegibilidad}

\begin{Shaded}
\begin{Highlighting}[]
\NormalTok{moda\_ciudades }\OtherTok{\textless{}{-}}\NormalTok{ ciudad\_elegibilidad }\SpecialCharTok{\%\textgreater{}\%}
  \FunctionTok{group\_by}\NormalTok{(Clean.Alternative.Fuel.Vehicle..CAFV..Eligibility) }\SpecialCharTok{\%\textgreater{}\%}
  \FunctionTok{slice\_max}\NormalTok{(Conteo, }\AttributeTok{n =} \DecValTok{5}\NormalTok{)  }\CommentTok{\# Tomar las 5 ciudades con más registros}
\CommentTok{\# Tomar las 5 ciudades con más registros}
\end{Highlighting}
\end{Shaded}

\subsubsection{Obtener la mediana del conteo de registros por
ciudad}\label{obtener-la-mediana-del-conteo-de-registros-por-ciudad}

\begin{Shaded}
\begin{Highlighting}[]
\NormalTok{mediana\_conteo }\OtherTok{\textless{}{-}} \FunctionTok{median}\NormalTok{(ciudad\_elegibilidad}\SpecialCharTok{$}\NormalTok{Conteo)}
\end{Highlighting}
\end{Shaded}

\subsubsection{Obtener la media del conteo de registros por
ciudad}\label{obtener-la-media-del-conteo-de-registros-por-ciudad}

\begin{Shaded}
\begin{Highlighting}[]
\NormalTok{media\_conteo }\OtherTok{\textless{}{-}} \FunctionTok{mean}\NormalTok{(ciudad\_elegibilidad}\SpecialCharTok{$}\NormalTok{Conteo)}
\end{Highlighting}
\end{Shaded}

\subsubsection{Mostrar resultados}\label{mostrar-resultados}

\begin{Shaded}
\begin{Highlighting}[]
\FunctionTok{print}\NormalTok{(}\StringTok{"Moda (Top ciudades por elegibilidad):"}\NormalTok{) }
\end{Highlighting}
\end{Shaded}

\begin{verbatim}
## [1] "Moda (Top ciudades por elegibilidad):"
\end{verbatim}

\begin{Shaded}
\begin{Highlighting}[]
\FunctionTok{print}\NormalTok{(moda\_ciudades)}
\end{Highlighting}
\end{Shaded}

\begin{verbatim}
## # A tibble: 15 x 3
## # Groups:   Clean.Alternative.Fuel.Vehicle..CAFV..Eligibility [3]
##    City      Clean.Alternative.Fuel.Vehicle..CAFV..Eligibility            Conteo
##    <chr>     <chr>                                                         <int>
##  1 Seattle   Clean Alternative Fuel Vehicle Eligible                       12269
##  2 Bellevue  Clean Alternative Fuel Vehicle Eligible                        3136
##  3 Vancouver Clean Alternative Fuel Vehicle Eligible                        2840
##  4 Redmond   Clean Alternative Fuel Vehicle Eligible                        2180
##  5 Renton    Clean Alternative Fuel Vehicle Eligible                        2078
##  6 Seattle   Eligibility unknown as battery range has not been researched  21596
##  7 Bellevue  Eligibility unknown as battery range has not been researched   7690
##  8 Redmond   Eligibility unknown as battery range has not been researched   5565
##  9 Bothell   Eligibility unknown as battery range has not been researched   5490
## 10 Vancouver Eligibility unknown as battery range has not been researched   4666
## 11 Seattle   Not eligible due to low battery range                          3545
## 12 Vancouver Not eligible due to low battery range                          1004
## 13 Tukwila   Not eligible due to low battery range                           826
## 14 Renton    Not eligible due to low battery range                           709
## 15 Bellevue  Not eligible due to low battery range                           683
\end{verbatim}

\begin{Shaded}
\begin{Highlighting}[]
\FunctionTok{print}\NormalTok{(}\FunctionTok{paste}\NormalTok{(}\StringTok{"Mediana de registros por ciudad:"}\NormalTok{, mediana\_conteo))}
\end{Highlighting}
\end{Shaded}

\begin{verbatim}
## [1] "Mediana de registros por ciudad: 6"
\end{verbatim}

\begin{Shaded}
\begin{Highlighting}[]
\FunctionTok{print}\NormalTok{(}\FunctionTok{paste}\NormalTok{(}\StringTok{"Media de registros por ciudad:"}\NormalTok{, }\FunctionTok{round}\NormalTok{(media\_conteo, }\DecValTok{2}\NormalTok{)))}
\end{Highlighting}
\end{Shaded}

\begin{verbatim}
## [1] "Media de registros por ciudad: 145.13"
\end{verbatim}

\begin{Shaded}
\begin{Highlighting}[]
\CommentTok{\#windows()}
\FunctionTok{ggplot}\NormalTok{(moda\_ciudades, }\FunctionTok{aes}\NormalTok{(}\AttributeTok{x =} \FunctionTok{reorder}\NormalTok{(City, Conteo), }\AttributeTok{y =}
\NormalTok{Conteo, }\AttributeTok{fill =}\NormalTok{ Clean.Alternative.Fuel.Vehicle..CAFV..Eligibility)) }\SpecialCharTok{+}
\FunctionTok{geom\_bar}\NormalTok{(}\AttributeTok{stat =} \StringTok{"identity"}\NormalTok{, }\AttributeTok{position =} \StringTok{"dodge"}\NormalTok{) }\SpecialCharTok{+} \FunctionTok{coord\_flip}\NormalTok{() }\SpecialCharTok{+}
\FunctionTok{labs}\NormalTok{(}\AttributeTok{title =} \StringTok{"Ciudades con Mayor Elegibilidad CAFV"}\NormalTok{, }\AttributeTok{x =} \StringTok{"Ciudad"}\NormalTok{, }\AttributeTok{y =}
\StringTok{"Cantidad de Vehículos"}\NormalTok{) }\SpecialCharTok{+} \FunctionTok{theme\_minimal}\NormalTok{()}
\end{Highlighting}
\end{Shaded}

\includegraphics{ProyectoPrimerCorte_files/figure-latex/unnamed-chunk-12-1.pdf}

\subsubsection{Pregunta 5: Relación entre tipo de vehículo y proveedor
eléctrico}\label{pregunta-5-relaciuxf3n-entre-tipo-de-vehuxedculo-y-proveedor-eluxe9ctrico}

\subsubsection{Contar registros por Tipo de Vehículo y Proveedor
Eléctrico}\label{contar-registros-por-tipo-de-vehuxedculo-y-proveedor-eluxe9ctrico}

\begin{Shaded}
\begin{Highlighting}[]
\NormalTok{vehiculo\_proveedor }\OtherTok{\textless{}{-}}\NormalTok{ datos }\SpecialCharTok{\%\textgreater{}\%}
  \FunctionTok{group\_by}\NormalTok{(Electric.Vehicle.Type, Electric.Utility) }\SpecialCharTok{\%\textgreater{}\%}
  \FunctionTok{summarise}\NormalTok{(}\AttributeTok{Conteo =} \FunctionTok{n}\NormalTok{(), }\AttributeTok{.groups =} \StringTok{"drop"}\NormalTok{) }\SpecialCharTok{\%\textgreater{}\%}
  \FunctionTok{arrange}\NormalTok{(}\FunctionTok{desc}\NormalTok{(Conteo))}
\end{Highlighting}
\end{Shaded}

\subsubsection{Obtener la moda (las combinaciones más
frecuentes)}\label{obtener-la-moda-las-combinaciones-muxe1s-frecuentes}

\begin{Shaded}
\begin{Highlighting}[]
\NormalTok{moda\_vehiculo\_proveedor }\OtherTok{\textless{}{-}}\NormalTok{ vehiculo\_proveedor }\SpecialCharTok{\%\textgreater{}\%}
  \FunctionTok{group\_by}\NormalTok{(Electric.Vehicle.Type) }\SpecialCharTok{\%\textgreater{}\%}
  \FunctionTok{slice\_max}\NormalTok{(Conteo, }\AttributeTok{n =} \DecValTok{5}\NormalTok{)  }\CommentTok{\# Tomar las 5 combinaciones más comunes}
\end{Highlighting}
\end{Shaded}

\subsubsection{Calcular mediana y media del conteo de
registros}\label{calcular-mediana-y-media-del-conteo-de-registros}

\begin{Shaded}
\begin{Highlighting}[]
\NormalTok{mediana\_conteo }\OtherTok{\textless{}{-}} \FunctionTok{median}\NormalTok{(vehiculo\_proveedor}\SpecialCharTok{$}\NormalTok{Conteo)}
\NormalTok{media\_conteo }\OtherTok{\textless{}{-}} \FunctionTok{mean}\NormalTok{(vehiculo\_proveedor}\SpecialCharTok{$}\NormalTok{Conteo)}
\end{Highlighting}
\end{Shaded}

\subsubsection{Mostrar resultados en
consola}\label{mostrar-resultados-en-consola}

\begin{Shaded}
\begin{Highlighting}[]
\FunctionTok{print}\NormalTok{(}\StringTok{"Moda (Top combinaciones de Tipo de Vehículo y Proveedor Eléctrico):"}\NormalTok{)}
\end{Highlighting}
\end{Shaded}

\begin{verbatim}
## [1] "Moda (Top combinaciones de Tipo de Vehículo y Proveedor Eléctrico):"
\end{verbatim}

\begin{Shaded}
\begin{Highlighting}[]
\FunctionTok{print}\NormalTok{(moda\_vehiculo\_proveedor)}
\end{Highlighting}
\end{Shaded}

\begin{verbatim}
## # A tibble: 10 x 3
## # Groups:   Electric.Vehicle.Type [2]
##    Electric.Vehicle.Type                  Electric.Utility                Conteo
##    <chr>                                  <chr>                            <int>
##  1 Battery Electric Vehicle (BEV)         PUGET SOUND ENERGY INC||CITY O~  69806
##  2 Battery Electric Vehicle (BEV)         PUGET SOUND ENERGY INC           39356
##  3 Battery Electric Vehicle (BEV)         CITY OF SEATTLE - (WA)|CITY OF~  32278
##  4 Battery Electric Vehicle (BEV)         BONNEVILLE POWER ADMINISTRATIO~  10361
##  5 Battery Electric Vehicle (BEV)         BONNEVILLE POWER ADMINISTRATIO~   8271
##  6 Plug-in Hybrid Electric Vehicle (PHEV) PUGET SOUND ENERGY INC||CITY O~  15492
##  7 Plug-in Hybrid Electric Vehicle (PHEV) PUGET SOUND ENERGY INC            9567
##  8 Plug-in Hybrid Electric Vehicle (PHEV) CITY OF SEATTLE - (WA)|CITY OF~   7828
##  9 Plug-in Hybrid Electric Vehicle (PHEV) BONNEVILLE POWER ADMINISTRATIO~   3400
## 10 Plug-in Hybrid Electric Vehicle (PHEV) BONNEVILLE POWER ADMINISTRATIO~   2414
\end{verbatim}

\begin{Shaded}
\begin{Highlighting}[]
\FunctionTok{print}\NormalTok{(}\FunctionTok{paste}\NormalTok{(}\StringTok{"Mediana de registros por combinación:"}\NormalTok{, mediana\_conteo))}
\end{Highlighting}
\end{Shaded}

\begin{verbatim}
## [1] "Mediana de registros por combinación: 108"
\end{verbatim}

\begin{Shaded}
\begin{Highlighting}[]
\FunctionTok{print}\NormalTok{(}\FunctionTok{paste}\NormalTok{(}\StringTok{"Media de registros por combinación:"}\NormalTok{, }\FunctionTok{round}\NormalTok{(media\_conteo, }\DecValTok{2}\NormalTok{)))}
\end{Highlighting}
\end{Shaded}

\begin{verbatim}
## [1] "Media de registros por combinación: 1581.83"
\end{verbatim}

\subsubsection{🔹 Visualización de las combinaciones más
comunes}\label{visualizaciuxf3n-de-las-combinaciones-muxe1s-comunes}

\begin{Shaded}
\begin{Highlighting}[]
\CommentTok{\#windows()}
\FunctionTok{ggplot}\NormalTok{(moda\_vehiculo\_proveedor, }\FunctionTok{aes}\NormalTok{(}\AttributeTok{x =} \FunctionTok{reorder}\NormalTok{(Electric.Utility, Conteo), }\AttributeTok{y =}\NormalTok{ Conteo, }\AttributeTok{fill =}\NormalTok{ Electric.Vehicle.Type)) }\SpecialCharTok{+}
  \FunctionTok{geom\_bar}\NormalTok{(}\AttributeTok{stat =} \StringTok{"identity"}\NormalTok{, }\AttributeTok{position =} \StringTok{"dodge"}\NormalTok{) }\SpecialCharTok{+}
  \FunctionTok{coord\_flip}\NormalTok{() }\SpecialCharTok{+}
  \FunctionTok{labs}\NormalTok{(}\AttributeTok{title =} \StringTok{"Relación entre Tipo de Vehículo y Proveedor Eléctrico"}\NormalTok{,}
       \AttributeTok{x =} \StringTok{"Proveedor Eléctrico"}\NormalTok{,}
       \AttributeTok{y =} \StringTok{"Cantidad de Vehículos"}\NormalTok{) }\SpecialCharTok{+}
  \FunctionTok{theme\_minimal}\NormalTok{()}
\end{Highlighting}
\end{Shaded}

\includegraphics{ProyectoPrimerCorte_files/figure-latex/unnamed-chunk-17-1.pdf}

\paragraph{\texorpdfstring{\textbf{Conclusiones}}{Conclusiones}}\label{conclusiones}

\begin{itemize}
\item
  Los lugares donde los proveedores de electricidad han desarrollado una
  buena red de carga y existen incentivos claros son los que tienen más
  vehículos eléctricos. Como nos demuestran las gráficas de la ciudad de
  Seattle.
\item
  Aunque los híbridos enchufables (PHEV) se encuentren en el mercado, la
  mayoría de las personas están optando por los vehículos 100\%
  eléctricos (BEV). Esto indica que la tecnología ha avanzado
  permitiendo mayor autonomía en estos, que los incentivos y lugares de
  carga cada vez están mejor distribuidos.
\item
  Las empresas de electricidad tienen un rol clave en el futuro de la
  movilidad. No es solo una cuestión de que las personas quieran cambiar
  a autos eléctricos, sino que las compañías eléctricas deben estar
  preparadas para abastecer esa demanda. Aquellas que ya han trabajado
  en infraestructura están viendo cómo en sus regiones el número de
  autos eléctricos crece más rápido.
\item
  Las compañías eléctricas deben estar preparadas para abastecer la
  demanda con su infraestructura ya que~ están viendo cómo en sus
  regiones el número de autos eléctricos crece más rápido.
\item
  Aunque hay avances importantes, aún hay proveedores con muy pocos
  autos eléctricos en sus redes, lo que indica que en algunas zonas
  falta infraestructura o incentivos atractivos.
\end{itemize}

\paragraph{\texorpdfstring{\textbf{Referencias}}{Referencias}}\label{referencias}

\begin{itemize}
\tightlist
\item
  Data.gov. (n.d.). Electric Vehicle Population Data. Retrieved
  from\url{https://catalog.data.gov/dataset/electric-vehicle-population-data}
\end{itemize}

\end{document}
